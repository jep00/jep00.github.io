\title{Full Education Profile}\author{Joseph Pym}\date{Last Updated: \today}
\documentclass[9pt,a4paper]{article}
\usepackage{amsmath} \usepackage{amssymb} \usepackage{wrapfig}
\usepackage{fancyvrb} \usepackage{appendix} \usepackage{graphicx}
\usepackage{caption,subcaption} \usepackage{hyperref} \usepackage{arydshln}
\begin{document}
\maketitle
\tableofcontents\pagebreak

%\section{University of Leeds}
%\begin{center}
%	\includegraphics[height=0.1\textheight]{uollogo.png}\\\quad\\
%	\includegraphics[height=0.15\textheight]{uolcrest.png}
%\end{center}
%\begin{center}
%	\noindent\textbf{Future}\\
%	\noindent\textbf{MSc Data Science and Analytics}\\
%	\noindent from the University of Leeds.\\
%	\noindent\emph{ 2021 \textemdash\ 2022 (exp.)}
%\end{center}

%\pagebreak
\section{Bachelor's Degree: Coventry University}
\begin{center}
	\includegraphics[height=0.15\textheight]{covuni.jpg}\\\quad\\
	\includegraphics[height=0.15\textheight]{covuni_coa.png}
\end{center}
\begin{center}
\noindent\textbf{First Class Degree}\\
\noindent\textbf{BSc (Hons.) Mathematics and Statistics}\\
\noindent from Coventry University.\\
\noindent\emph{ 2018 \textemdash\ 2021}
\end{center}

Completed a Bachelor's Degree with Honours at Coventry University, studying Mathematics and Statistics in the School of Computing, Engineering \& Mathematics in the Engineering, Environment \& Computing faculty.\\\linebreak
120 credits were required to pass each year, with the majority of course modules counting for 20; the Professional \& Academic skills modules in the first and second years, and the Advanced Linear Algebra module in the third year counted for 10 credits; and a yearly `Addvantage' ($\dagger$) module, unrelated to the core course content, counted for 10 credits. In the final year, the final year project (dissertation) counted for 30 credits.\\\linebreak
\emph{Gained experience using the following programs}: R, Python, MATLAB, Microsoft Excel, SPSS, Simulink, Minitab.\\ \linebreak
\emph{Third year project}: titled `Assessing the Accuracy of Betting Odds in European Football,' supervised by Dr. Maha Moustafa and Dr. Alun Owen. 

\pagebreak

\noindent
\begin{tabular}{p{9cm} | r}
	\emph{Module} & \emph{Grade (\%)} \\ \hline\hline
	\textbf{First Year} & \\\hline
	Algebra 											& 88.20\\
	Calculus											& 80.90\\
	Mathematical Modelling (Mechanics)					& 80.80\\
	Mathematical \& Numerical Analysis					& 77.85\\
	Statistics											& 75.10\\
	\hdashline
	Professional \& Academic Skills I					& 87.00\\
	Introduction to Book-keeping $\dagger$				& 79.00\\
	\hline\hline
	\textbf{Second Year} & \\\hline
	Discrete Mathematics								&100.00\\
	Statistical Computing								& 87.74\\
	Further Calculus \& Algebra							& 78.61\\
	Operational Research Techniques						& 75.00\\
	Linear Statistical Models							& 69.60\\
	\hdashline
	Professional \& Academic Skills II					& 74.33\\
	MATLAB \& Simulink for Research in Industry $\dagger$ & 55.00\\
	\hline\hline
	\textbf{Third Year} & \\\hline
	Project												& 77.00\\
	\hdashline
	Optimisation										& 97.25\\
	Advanced Topics in Statistics						& 90.50\\
	Coding \& Cryptography								& 78.20\\
	Statistical Modelling								& 66.20\\
	\hdashline
	Advanced Linear Algebra and its Applications		& 78.50\\
	Students' Union Work Experience						& 70.00
	
\end{tabular}


\pagebreak
\section{A Levels: Hutton Grammar Sixth Form}
\begin{center}
	\includegraphics[height=0.15\textheight]{hgs.jpg}
\end{center}
\begin{center}
	\noindent\textbf{A---Mathematics; \\A---Chemistry; \\C---Physics.}\\
	\noindent from Hutton Grammar Sixth Form.\\
	\noindent\emph{2016---2018}
\end{center}

Mathematics studied with the statistics (S1, S2) modules alongside the core modules (C1, C2, C3, C4)\\\linebreak
Also completed the Extended Project Qualification with the topic title ``Will there be supersonic transport again by 2030, and if so will it be sustainable?'', which involved a 5,000 word report and 15 minute presentation with questions, achieving an A.
\\\linebreak
\noindent
\begin{center}\begin{tabular}{p{3.2cm}|c|p{4.5cm}}
	\emph{Subject}&\emph{Grade}&\emph{Notes}\\\hline\hline
	Mathematics & A & With Statistics modules \\
	Chemistry 	& A & With Practical Endorsement\\
	Physics 	& C & With Practical Endorsement
\end{tabular}\end{center}

\pagebreak


\section{GCSEs: Hutton Grammar School}

\begin{center}
	\includegraphics[height=0.15\textheight]{hgs.jpg}
\end{center}
\begin{center}
	\noindent\textbf{3 A*, 6 A, 4 B}\\
	\noindent\textit{Including Mathematics (A*), English Language (A), English Literature.}\\
	\noindent from Hutton Grammar School.\\
	\noindent\emph{2011---2016}
\end{center}


\begin{center}\begin{tabular}{p{4.2cm}|c|p{4.5cm}}
	\emph{Subject}&\emph{Grade}&\emph{Notes}\\\hline\hline
	Mathematics & A* & \\
	I.C.T. & 2 A*'s &TLM GCSE Equivalents$\dagger$\\
	English Literature& A &\\
	English Language&A&\\
	Biology&A&\\
	Computing& A &\\
	Chemistry & A &\\
	Physics&A&\\
	Further Mathematics&B&\\
	German&B&\\
	Geography&B&\\
	Religious Studies&B&
\end{tabular}\end{center}

\emph{$\dagger$The TLM grades were Distinctions, equivalent to a GCSE A*.}

\pagebreak
\section{Other Qualifications}
\textbf{Duke of Edinburgh Award}\\
I was a member of the Air Training Corps (Air Cadets) for four years, during which time I achieved the \underline{Bronze Duke of Edinburgh Award}, which taught me a lot about working as both a team member and leader, verbal communication, perseverance and following instructions.\\\linebreak
\textbf{BTEC Aviation Studies}\\
In addition to the DofE Award, I also studied for a BTEC in \underline{Aviation Studies} whilst at Air Cadets. Once I had achieved the qualification (at a Pass level; equivalent to four GCSEs), I then went on the \emph{Method of Instruction} course, allowing me to instruct newer cadets.
\end{document}