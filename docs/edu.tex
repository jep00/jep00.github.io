\title{Full Education Profile}\author{Joseph Pym}\date{Last Updated: \today}
\documentclass[9pt,a4paper]{article}
\usepackage{amsmath} \usepackage{amssymb} \usepackage{wrapfig}
\usepackage{fancyvrb} \usepackage{appendix} \usepackage{graphicx}
\usepackage{caption,subcaption} \usepackage{hyperref} \usepackage{arydshln}
\begin{document}
\maketitle
\tableofcontents\pagebreak
\section{Coventry University}
\begin{center}
	\includegraphics[height=0.15\textheight]{covuni.jpg}
	\includegraphics[height=0.15\textheight]{covuni_coa.png}
\end{center}
\begin{center}
\noindent\textbf{Ongoing}, expected grade: First Class\\
\noindent\textbf{BSc Mathematics and Statistics}\\
\noindent from Coventry University.\\
\noindent\emph{September 2018 \textemdash\ 2021 (exp.)}
\end{center}

Currently a student at Coventry University, studying Mathematics and Statistics in the School of Computing, Engineering \& Mathematics in the Engineering, Environment \& Computing faculty. \\ \linebreak
120 credits are required to pass each year, with course modules counting for 20. The Professional \& Academic skills modules in the first and second years, and the Advanced Linear Algebra module in the third year counted for 10 credits; I also had to complete a yearly `Addvantage' ($\dagger$) module, unrelated to my core course content, worth 10 credits.\\\linebreak
\underline{First Year modules}: \\
\begin{tabular}{p{6.2cm}|p{1.1cm}|p{3.5cm}}
	\emph{Module}&\emph{Grade}&\emph{Notes}\\\hline\hline
	101MP: Algebra&88.20\%&\\\hdashline
	102MP: Calculus&80.90\%&\\\hdashline
	130MP: Professional \& Academic Skills&87.00\%&Use of MATLAB\\\hdashline
	A169ACC: Introduction to Book-Keeping&79.00\%&$\dagger$\\\hdashline
	103MP: Mathematical and Numerical Analysis&77.85\%&Use of MATLAB\\\hdashline
	110MP: Mathematical Modelling&80.80\%&(Mechanics); Use of Excel\\\hdashline
	120MP: Statistics&75.10\%&Use of Minitab\\\hline
	\textbf{Year One Average:}&\textbf{80.85\%}
\end{tabular}
\pagebreak


\noindent\underline{Second Year modules}: \\
\begin{tabular}{p{6.2cm}|p{1.1cm}|p{3.5cm}}
	\emph{Module}						&\emph{Grade}	&\emph{Notes}\\\hline\hline
	201MP: Further Algebra \& Calculus	&78.61\%		&\\\hdashline
	221MP: Statistical Computing		&87.74\%		&Use of SPSS and R\\\hdashline
	230MP: Professional \& Academic Skills&74.33\%		&Use of Python\\\hdashline
	A241MS: MATLAB and Simulink for Research in Industry&55.00\%&$\dagger$ Use of MATLAB and Simulink\\\hdashline
	201CEM: Operational Research Techniques&75.00\%	&Use of Excel\\\hdashline
	203MP: Discrete Mathematics			&100\%			&Graph Theory and Stochastic Statistics\\\hdashline
	220MP: Linear Statistical Models	&69.60\%&Use of Minitab\\\hline
	\textbf{Year Two Average:}&\textbf{79.27\%}
\end{tabular}
\vspace{7mm}

\noindent\underline{Third Year modules}: \\
\begin{tabular}{p{6.2cm}|p{1.1cm}|p{3.5cm}}
	\emph{Module}						&\emph{Grade}	&\emph{Notes}\\\hline\hline
	3005CEM: Advanced Linear Algebra	&78.50\%		&\\\hdashline
	302MP: Coding \& Cryptography		&78.20\%		&Use of Python and MATLAB\\\hdashline
	320MP: Statistical Design \& Modelling&66.20\%		&Use of R and Minitab\\\hdashline
	A306CSU: Students' Union Work Experience&70.00\%&$\dagger$\\\hdashline
	321MP: Advanced Topics in Statistics	& --- &Use of R and Excel\\\hdashline
	322MP: Optimisation		&---			&Use of Python\\\hdashline
	331MP: \textit{Project}				& ---		&Use of R\\\hline
	\textbf{Year Three Average:}&\textbf{---}& \\ \hline
	\textbf{Final University Average:} & ---
\end{tabular}
\vspace{7mm}

\noindent\textit{Third year project: }\\``Assessing the Accuracy of Betting Odds in European Football''\\
Using R for the coding, with dissertation wrote in \LaTeX\ submitted.

\pagebreak
\section{A Levels: Hutton Grammar Sixth Form}
\begin{center}
	\includegraphics[height=0.15\textheight]{hgs.jpg}
\end{center}
\begin{center}
	\noindent\textbf{A---Mathematics; \\A---Chemistry; \\C---Physics}\\
	\noindent from Hutton Grammar Sixth Form.\\
	\noindent\emph{2016---2018}
\end{center}

Mathematics studied with the statistics (S1, S2) modules alongside the core modules (C1, C2, C3, C4)\\
Also completed the Extended Project Qualification with the topic title ``Will there be supersonic transport again by 2030, and if so will it be sustainable?'', which involved a 5,000 word report and 15 minute presentation with questions, achieving an A.
 \\\linebreak
\underline{Final Grades:}\\
\begin{tabular}{p{5.2cm}|p{1.1cm}|p{4.5cm}}
	\emph{Subject}&\emph{Grade}&\emph{Notes}\\\hline\hline
	Mathematics and Statistics & A & \\
	Chemistry & A &With Practical Endorsement\\
	Physics & C &With Practical Endorsement \\
	Extended Project Qualification & A & Half UCAS Points\\
	General Studies & C & Half UCAS Points
\end{tabular}

\pagebreak


\section{GCSEs: Hutton Grammar School}

\begin{center}
	\includegraphics[height=0.15\textheight]{hgs.jpg}
\end{center}
\begin{center}
	\noindent\textbf{3 A*, 6 A, 4 B}\\
	\noindent\textit{Including Mathematics (A*), English Language (A), English Literature.}\\
	\noindent from Hutton Grammar School.\\
	\noindent\emph{2011---2016}
\end{center}

\underline{Final Grades:}\\
\begin{tabular}{p{5.2cm}|p{1.1cm}|p{4.5cm}}
	\emph{Subject}&\emph{Grade}&\emph{Notes}\\\hline\hline
	Mathematics & A* & \\
	I.C.T. & 2 A*'s &TLM GCSE Equivalents$\dagger$\\
	English Literature& A &\\
	English Language&A&\\
	Biology&A&\\
	Computing& A &\\
	Chemistry & A &\\
	Physics&A&\\
	Further Mathematics&B&\\
	German&B&\\
	Geography&B&\\
	Religious Studies&B&
\end{tabular}
\\\linebreak
\emph{$\dagger$The TLM grades were Distinctions, equivalent to a GCSE A*.}

\pagebreak
\section{Other Qualifications}
\textbf{Duke of Edinburgh Award}\\
I was a member of the Air Training Corps (Air Cadets) for four years, during which time I achieved the \underline{Bronze Duke of Edinburgh Award}, which taught me a lot about working as both a team member and leader, verbal communication, perseverance and following instructions.\\\linebreak
\textbf{BTEC Aviation Studies}\\
In addition to the DofE Award, I also studied for a BTEC in \underline{Aviation Studies} whilst at Air Cadets. Once I had achieved the qualification (at a Pass level; equivalent to four GCSEs), I then went on the \emph{Method of Instruction} course, allowing me to instruct newer cadets.
\end{document}